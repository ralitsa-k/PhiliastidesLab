% Options for packages loaded elsewhere
\PassOptionsToPackage{unicode}{hyperref}
\PassOptionsToPackage{hyphens}{url}
%
\documentclass[
]{book}
\usepackage{lmodern}
\usepackage{amssymb,amsmath}
\usepackage{ifxetex,ifluatex}
\ifnum 0\ifxetex 1\fi\ifluatex 1\fi=0 % if pdftex
  \usepackage[T1]{fontenc}
  \usepackage[utf8]{inputenc}
  \usepackage{textcomp} % provide euro and other symbols
\else % if luatex or xetex
  \usepackage{unicode-math}
  \defaultfontfeatures{Scale=MatchLowercase}
  \defaultfontfeatures[\rmfamily]{Ligatures=TeX,Scale=1}
\fi
% Use upquote if available, for straight quotes in verbatim environments
\IfFileExists{upquote.sty}{\usepackage{upquote}}{}
\IfFileExists{microtype.sty}{% use microtype if available
  \usepackage[]{microtype}
  \UseMicrotypeSet[protrusion]{basicmath} % disable protrusion for tt fonts
}{}
\makeatletter
\@ifundefined{KOMAClassName}{% if non-KOMA class
  \IfFileExists{parskip.sty}{%
    \usepackage{parskip}
  }{% else
    \setlength{\parindent}{0pt}
    \setlength{\parskip}{6pt plus 2pt minus 1pt}}
}{% if KOMA class
  \KOMAoptions{parskip=half}}
\makeatother
\usepackage{xcolor}
\IfFileExists{xurl.sty}{\usepackage{xurl}}{} % add URL line breaks if available
\IfFileExists{bookmark.sty}{\usepackage{bookmark}}{\usepackage{hyperref}}
\hypersetup{
  pdftitle={241 Lab Book},
  pdfauthor={Ralitsa Kostova},
  hidelinks,
  pdfcreator={LaTeX via pandoc}}
\urlstyle{same} % disable monospaced font for URLs
\usepackage{color}
\usepackage{fancyvrb}
\newcommand{\VerbBar}{|}
\newcommand{\VERB}{\Verb[commandchars=\\\{\}]}
\DefineVerbatimEnvironment{Highlighting}{Verbatim}{commandchars=\\\{\}}
% Add ',fontsize=\small' for more characters per line
\usepackage{framed}
\definecolor{shadecolor}{RGB}{248,248,248}
\newenvironment{Shaded}{\begin{snugshade}}{\end{snugshade}}
\newcommand{\AlertTok}[1]{\textcolor[rgb]{0.94,0.16,0.16}{#1}}
\newcommand{\AnnotationTok}[1]{\textcolor[rgb]{0.56,0.35,0.01}{\textbf{\textit{#1}}}}
\newcommand{\AttributeTok}[1]{\textcolor[rgb]{0.77,0.63,0.00}{#1}}
\newcommand{\BaseNTok}[1]{\textcolor[rgb]{0.00,0.00,0.81}{#1}}
\newcommand{\BuiltInTok}[1]{#1}
\newcommand{\CharTok}[1]{\textcolor[rgb]{0.31,0.60,0.02}{#1}}
\newcommand{\CommentTok}[1]{\textcolor[rgb]{0.56,0.35,0.01}{\textit{#1}}}
\newcommand{\CommentVarTok}[1]{\textcolor[rgb]{0.56,0.35,0.01}{\textbf{\textit{#1}}}}
\newcommand{\ConstantTok}[1]{\textcolor[rgb]{0.00,0.00,0.00}{#1}}
\newcommand{\ControlFlowTok}[1]{\textcolor[rgb]{0.13,0.29,0.53}{\textbf{#1}}}
\newcommand{\DataTypeTok}[1]{\textcolor[rgb]{0.13,0.29,0.53}{#1}}
\newcommand{\DecValTok}[1]{\textcolor[rgb]{0.00,0.00,0.81}{#1}}
\newcommand{\DocumentationTok}[1]{\textcolor[rgb]{0.56,0.35,0.01}{\textbf{\textit{#1}}}}
\newcommand{\ErrorTok}[1]{\textcolor[rgb]{0.64,0.00,0.00}{\textbf{#1}}}
\newcommand{\ExtensionTok}[1]{#1}
\newcommand{\FloatTok}[1]{\textcolor[rgb]{0.00,0.00,0.81}{#1}}
\newcommand{\FunctionTok}[1]{\textcolor[rgb]{0.00,0.00,0.00}{#1}}
\newcommand{\ImportTok}[1]{#1}
\newcommand{\InformationTok}[1]{\textcolor[rgb]{0.56,0.35,0.01}{\textbf{\textit{#1}}}}
\newcommand{\KeywordTok}[1]{\textcolor[rgb]{0.13,0.29,0.53}{\textbf{#1}}}
\newcommand{\NormalTok}[1]{#1}
\newcommand{\OperatorTok}[1]{\textcolor[rgb]{0.81,0.36,0.00}{\textbf{#1}}}
\newcommand{\OtherTok}[1]{\textcolor[rgb]{0.56,0.35,0.01}{#1}}
\newcommand{\PreprocessorTok}[1]{\textcolor[rgb]{0.56,0.35,0.01}{\textit{#1}}}
\newcommand{\RegionMarkerTok}[1]{#1}
\newcommand{\SpecialCharTok}[1]{\textcolor[rgb]{0.00,0.00,0.00}{#1}}
\newcommand{\SpecialStringTok}[1]{\textcolor[rgb]{0.31,0.60,0.02}{#1}}
\newcommand{\StringTok}[1]{\textcolor[rgb]{0.31,0.60,0.02}{#1}}
\newcommand{\VariableTok}[1]{\textcolor[rgb]{0.00,0.00,0.00}{#1}}
\newcommand{\VerbatimStringTok}[1]{\textcolor[rgb]{0.31,0.60,0.02}{#1}}
\newcommand{\WarningTok}[1]{\textcolor[rgb]{0.56,0.35,0.01}{\textbf{\textit{#1}}}}
\usepackage{longtable,booktabs}
% Correct order of tables after \paragraph or \subparagraph
\usepackage{etoolbox}
\makeatletter
\patchcmd\longtable{\par}{\if@noskipsec\mbox{}\fi\par}{}{}
\makeatother
% Allow footnotes in longtable head/foot
\IfFileExists{footnotehyper.sty}{\usepackage{footnotehyper}}{\usepackage{footnote}}
\makesavenoteenv{longtable}
\usepackage{graphicx,grffile}
\makeatletter
\def\maxwidth{\ifdim\Gin@nat@width>\linewidth\linewidth\else\Gin@nat@width\fi}
\def\maxheight{\ifdim\Gin@nat@height>\textheight\textheight\else\Gin@nat@height\fi}
\makeatother
% Scale images if necessary, so that they will not overflow the page
% margins by default, and it is still possible to overwrite the defaults
% using explicit options in \includegraphics[width, height, ...]{}
\setkeys{Gin}{width=\maxwidth,height=\maxheight,keepaspectratio}
% Set default figure placement to htbp
\makeatletter
\def\fps@figure{htbp}
\makeatother
\setlength{\emergencystretch}{3em} % prevent overfull lines
\providecommand{\tightlist}{%
  \setlength{\itemsep}{0pt}\setlength{\parskip}{0pt}}
\setcounter{secnumdepth}{5}
\usepackage{booktabs}
\usepackage{amsthm}
\makeatletter
\def\thm@space@setup{%
  \thm@preskip=8pt plus 2pt minus 4pt
  \thm@postskip=\thm@preskip
}
\makeatother
\usepackage[]{natbib}
\bibliographystyle{apalike}

\title{241 Lab Book}
\author{Ralitsa Kostova}
\date{2021-03-11}

\begin{document}
\maketitle

{
\setcounter{tocdepth}{1}
\tableofcontents
}
\begin{Shaded}
\begin{Highlighting}[]
\KeywordTok{install.packages}\NormalTok{(}\StringTok{"bookdown"}\NormalTok{)}
\CommentTok{# or the development version}
\CommentTok{# devtools::install_github("rstudio/bookdown")}
\end{Highlighting}
\end{Shaded}

Remember each Rmd file contains one and only one chapter, and a chapter is defined by the first-level heading \texttt{\#}.

To compile this example to PDF, you need XeLaTeX. You are recommended to install TinyTeX (which includes XeLaTeX): \url{https://yihui.name/tinytex/}.

\hypertarget{intro}{%
\chapter{Introduction}\label{intro}}

You can label chapter and section titles using \texttt{\{\#label\}} after them, e.g., we can reference Chapter \ref{intro}. If you do not manually label them, there will be automatic labels anyway, e.g., Chapter \ref{methods}.

Figures and tables with captions will be placed in \texttt{figure} and \texttt{table} environments, respectively.

\begin{Shaded}
\begin{Highlighting}[]
\KeywordTok{par}\NormalTok{(}\DataTypeTok{mar =} \KeywordTok{c}\NormalTok{(}\DecValTok{4}\NormalTok{, }\DecValTok{4}\NormalTok{, }\FloatTok{.1}\NormalTok{, }\FloatTok{.1}\NormalTok{))}
\KeywordTok{plot}\NormalTok{(pressure, }\DataTypeTok{type =} \StringTok{'b'}\NormalTok{, }\DataTypeTok{pch =} \DecValTok{19}\NormalTok{)}
\end{Highlighting}
\end{Shaded}

\begin{figure}

{\centering \includegraphics[width=0.8\linewidth]{bookdown-demo_files/figure-latex/nice-fig-1} 

}

\caption{Here is a nice figure!}\label{fig:nice-fig}
\end{figure}

Reference a figure by its code chunk label with the \texttt{fig:} prefix, e.g., see Figure \ref{fig:nice-fig}. Similarly, you can reference tables generated from \texttt{knitr::kable()}, e.g., see Table \ref{tab:nice-tab}.

\begin{Shaded}
\begin{Highlighting}[]
\NormalTok{knitr}\OperatorTok{::}\KeywordTok{kable}\NormalTok{(}
  \KeywordTok{head}\NormalTok{(iris, }\DecValTok{20}\NormalTok{), }\DataTypeTok{caption =} \StringTok{'Here is a nice table!'}\NormalTok{,}
  \DataTypeTok{booktabs =} \OtherTok{TRUE}
\NormalTok{)}
\end{Highlighting}
\end{Shaded}

\begin{table}

\caption{\label{tab:nice-tab}Here is a nice table!}
\centering
\begin{tabular}[t]{rrrrl}
\toprule
Sepal.Length & Sepal.Width & Petal.Length & Petal.Width & Species\\
\midrule
5.1 & 3.5 & 1.4 & 0.2 & setosa\\
4.9 & 3.0 & 1.4 & 0.2 & setosa\\
4.7 & 3.2 & 1.3 & 0.2 & setosa\\
4.6 & 3.1 & 1.5 & 0.2 & setosa\\
5.0 & 3.6 & 1.4 & 0.2 & setosa\\
\addlinespace
5.4 & 3.9 & 1.7 & 0.4 & setosa\\
4.6 & 3.4 & 1.4 & 0.3 & setosa\\
5.0 & 3.4 & 1.5 & 0.2 & setosa\\
4.4 & 2.9 & 1.4 & 0.2 & setosa\\
4.9 & 3.1 & 1.5 & 0.1 & setosa\\
\addlinespace
5.4 & 3.7 & 1.5 & 0.2 & setosa\\
4.8 & 3.4 & 1.6 & 0.2 & setosa\\
4.8 & 3.0 & 1.4 & 0.1 & setosa\\
4.3 & 3.0 & 1.1 & 0.1 & setosa\\
5.8 & 4.0 & 1.2 & 0.2 & setosa\\
\addlinespace
5.7 & 4.4 & 1.5 & 0.4 & setosa\\
5.4 & 3.9 & 1.3 & 0.4 & setosa\\
5.1 & 3.5 & 1.4 & 0.3 & setosa\\
5.7 & 3.8 & 1.7 & 0.3 & setosa\\
5.1 & 3.8 & 1.5 & 0.3 & setosa\\
\bottomrule
\end{tabular}
\end{table}

You can write citations, too. For example, we are using the \textbf{bookdown} package \citep{R-bookdown} in this sample book, which was built on top of R Markdown and \textbf{knitr} \citep{xie2015}.

\hypertarget{EEG-setup}{%
\chapter{EEG-setup}\label{EEG-setup}}

Steps in short:

\begin{itemize}
\tightlist
\item
  Head size.
\item
  Cz position.
\item
  Cap on.
\item
  Alcohol wipe skin.
\item
  Attach 3 electrodes.
\item
  Full all with gel.
\item
  Rub all.
\item
  Fill again.
\item
  Go inside, attach to amps.
\item
  Green = Amp1, yellow = Amp2.
\item
  Amps on.
\item
  Check Ground, Ref, forehead (fix)
\item
  Check impedance at 0-100, then 0-50, and 0-20 scale.
\item
  Fix half head - fill with gel.
\item
  Fix other half - fill with gel.
\item
  Repeat rub and fill if necessary.
\item
  Show teeth clenching and eye squeeze.
\item
  Eye-calibration.
\item
  Save impedance.
\item
  Start task.
\end{itemize}

Steps detailed:

\begin{itemize}
\tightlist
\item
  Sit the participant on a chair outside the booth.
\item
  Measure head circumference just above the eyebrows.
\item
  Measure the position of Cz electrode.
\item
  Take the middle point between the nasion and the inion and mark it.
\item
  Take the middle point between the ears and align with the first point, make a bigger mark on the skin.
\item
  Use the cap with appropriate size and position by adjusting Cz, so that the mark is visible.
\item
  Pull the cap down tight, strap, and if necessary use the chest strap.
\item
  Clean the skin above the left eyebrow, just left of the chin, and behind the left ear with alcohol wipes.
\item
  Attach the stickers to the 3 loose electrodes and attach the electrodes to the cleaned areas.
\item
  Fill wach electrode with a syringe by holding pushin the electrodes close to the head, do not use too much gel.
\item
  Avoid connecting electrodes with hair strands.
\item
  Rub each hole with a cotton bud.
\item
  Fill each electrode once again.
\item
  Go inside the booth and attach the amp1 to the cable with green dot and amp2 to the cable with yellow.
\item
  Turn the amps on.
\item
  Check the ground, reference, and forehead impedance.
\item
  Go to Data and check the impedance of the rest of the electrodes by changing the scale from 0-100, then 0-50, and then 0-20
\item
  Aim for green/yellow dots by sideways motion (separates hair), or circling motion (rubs gel), or 360 degree motion (consistently improves impedance).
\item
  Show the participant the monitoring mode to explain what clencing the teeth, and squeezing the eyes does, and why it should be avoided.
\item
  Staying with their teeth slightly apart helps.
\item
  Explain what the blinks should be during the eye calibration, then play the eye calibration scenario
\item
  Save impedances by following the instructions by Brain Vision.
\item
  Do the experiment.
\end{itemize}

\hypertarget{cleaning-and-maintenance}{%
\chapter{Cleaning and maintenance}\label{cleaning-and-maintenance}}

\begin{itemize}
\tightlist
\item
  Clean the caps gently, very useful to use cotton pads, or simply the tap pressure to remove the gel.
\item
  Clean the working station.
\item
  Wipe the gel container.
\item
  If the gel is half empty or less, it is difficult to take it out, and it gets messy. Mix two gels in one container to make it easier to get the gel out.
\end{itemize}

\hypertarget{preprocessing}{%
\chapter{Preprocessing steps}\label{preprocessing}}

Ralie:
1. Run \textbf{events\_from\_csv.R} in Z:/TOOLS/RiPE\_soc/Beha\_check
This is the behavioural file which sends the events to the EEG machine. This step is to compare sent and received events later in Matlab.
This creates a csv file in each participant folder named dat\_all\_triggers.csv

\begin{enumerate}
\def\labelenumi{\arabic{enumi}.}
\setcounter{enumi}{1}
\item
  In EEG\_preprocess folder. Open \textbf{extract\_port\_events.m} Choose appropriate triggers, and rename the fields of the struct with \_onset, and \_id.
  This creates a .m files named `all\_events\_and\_behav\_socRiPE.m' (in my case).
\item
  Open \textbf{preprocess\_brainamp\_data.m} Fix the name of the .eeg file, initials of participants, etc. and run the script. This creates LP\_socRiPE\_filtered.mat (for example, for subject LP).
\item
  Run \textbf{Epoch data.} epoch = Should be defined as the name of the field of the structure that is of interest. faces\_onset in my case.
  This produces data epoched according to the offset values.
\item
\item
\end{enumerate}

\hypertarget{connection-to-the-grid}{%
\chapter{Connecting to the grid}\label{connection-to-the-grid}}

\begin{enumerate}
\def\labelenumi{\arabic{enumi}.}
\item
  Open terminal (on a mac) or a Putty session (on Windows: \url{https://www.chiark.greenend.org.uk/~sgtatham/putty/})
\item
  Type `ssh \href{mailto:yourusername@ccn00.psy.gla.ac.uk}{\nolinkurl{yourusername@ccn00.psy.gla.ac.uk}}' and enter your password when asked (on a mac) or connect to host `ccn00.psy.gla.ac.uk' with your user name and password on Putty (on Windows)
\end{enumerate}

Windows: connect to host ccn00.psy.gla.ac.uk
Terminal will prompt for account name: your user name
Then password: password
Then type vncserver
Make a new password.
Verify the password
Use
\url{https://www.tightvnc.com/download.php}

\begin{enumerate}
\def\labelenumi{\arabic{enumi}.}
\setcounter{enumi}{2}
\item
  Type `vncserver' in the terminal window (Putty should now put up a terminal window on Windows as well)
\item
  Make a note of your display number (e.g.~cn00.psy.gla.ac.uk:38)
\item
  Start your vnc client (chickenVNC for mac and tightVNC or realVNC for Windows - can be downloaded for free \url{https://www.tightvnc.com/download.php})
\item
  Type `ccn00' in the host box and your display number in the display box (for mac) or `ccn00:display number' in the server/host box (for Windows)
\item
  Enter your password
\item
  You should now see a desktop of the master node
\item
  Open a terminal in this new environment (the terminal option is in the Applications-System-Tools menu - in some instances a terminal shortcut might already exist in your navigation bar on the very top)
\item
  Connect to a free node (check node availability at ccn00.psy.gla.ac.uk/ganglia/) by typing `ssh compute-X-Y' where X, Y will correspond to the node ID from the link above
\item
  Set display environment by typing `setenv DISPLAY ccn00:43.0
\item
  To start matlab type `matlab \&'
\item
  To start FSL first type `source /usr/local/bin/fsl' then type `fsl \&'
\end{enumerate}

You can access our lab server in a terminal by typing: `cd /analyse/Project0130/'
Once you setup your jobs you can simply close the desktop window (and even shutdown your personal computer). This will not kill the jobs you have started on the grid. You can log back in to check the progress of your jobs by following steps 6-8 above while using the same display number as before.
To kill you vncsession all together (this will kill any jobs that might be running) simply type: `vncserver -kill :yourdisplaynumber'

\hypertarget{LDA}{%
\chapter{LDA process}\label{LDA}}

Some \emph{significant} applications are demonstrated in this chapter.

\hypertarget{example-one}{%
\section{Example one}\label{example-one}}

\hypertarget{example-two}{%
\section{Example two}\label{example-two}}

\hypertarget{remote-connection-to-office-computers}{%
\chapter{Remote conenction to office computers}\label{remote-connection-to-office-computers}}

\hypertarget{mac}{%
\section{Mac}\label{mac}}

\begin{enumerate}
\def\labelenumi{\arabic{enumi}.}
\item
  Download microsoft remote desktop client \url{https://apps.apple.com/us/app/microsoft-remote-desktop-10/id1295203466?mt=12}
\item
  Install and open the client.
\item
  Choose PCs.
\item
  Enter the following settings:
\end{enumerate}

\begin{itemize}
\tightlist
\item
  Connection name: your choice of nickname
\item
  PC name:
\item
  STAFF: ssdremote.gla.ac.uk
\item
  Username: campus\yourGUID (eg campus\abc1z)
\item
  Password: your GUID password
\item
  Full screen: deselect this if you want to be able to switch between the remote desktop and your Mac desktop more easily
\item
  All other settings: leave as default
\end{itemize}

\hypertarget{troubleshooting}{%
\chapter{Troubleshooting}\label{troubleshooting}}

If Syncbox connection error: USB reconnecting
- check that bottom right corner of Brainvision says PassiveCap64.
- if not passivecap64 \textgreater{} Open new project loading PassiveCap64.

  \bibliography{book.bib,packages.bib}

\end{document}
